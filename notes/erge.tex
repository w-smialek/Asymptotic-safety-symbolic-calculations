\documentclass[11pt, a4paper]{article} 
\usepackage[MeX]{polski} %polskie znaki
\usepackage{booktabs}
\usepackage{comment} 
\usepackage[utf8]{inputenc} % odpowiednie kodowanie znaków
\usepackage[T1, T2A]{fontenc} 
\usepackage{graphicx} %wstawianie obrazków
\usepackage{float} %ustawianie obraków/tabel
\usepackage{multirow}
\usepackage{amsmath, amsfonts,amsthm} 
\usepackage{amsthm} %do twierdzen itd
\usepackage{mathtools}
\usepackage{blindtext} 
%\usepackage{har2nat} %grzebie w bibliografii
\usepackage{geometry}
\usepackage{amssymb}
\usepackage{tikz-cd}
\usepackage{lscape}
\usepackage{hyperref}
\usepackage{XCharter}
\usepackage{dsfont}
\usepackage[labelfont=bf]{caption}
\usepackage{caption}
\usepackage{subcaption}

\setlength{\parindent}{15pt} 


\def\fR{\mathbb{R}}
\def\fC{\mathbb{C}}
\def\fN{\mathbb{N}}
\def\fZ{\mathbb{Z}}
%pogrubiony symbol wektora
\def\bv{\boldsymbol{v}}
\def\bu{\boldsymbol{u}}
\def\bR{\boldsymbol{R}}
\def\id{\mathds{1}}
\def\H{\mathcal{H}}
\def\idA{\vec{\mathbb{I}}_A{} }

\def\ra{\rangle}
\def\la{\langle}

%%%%%%%%%%%%%%%%% ME new 

\definecolor{dg}{rgb}{0,.5,0}

%%%%%%%%%%%%%%%%%%%%%%%%


\newtheorem{thm}{Theorem}[section]
\newtheorem{lem}[thm]{Lemma}
\newtheorem{corol}[thm]{Corollary}
\newtheorem{defn}[thm]{Definition}
\newtheorem{rmk}{Remark}

\title{\vspace{-2cm}The Functional Renormalization Group Equation}
% \author{Rafał Bistroń}
\date{}

\begin{document}

\maketitle

% \tableofcontents

\begin{center}
    \subsection*{The beta functional}
\end{center}

In the traditional Wilsonian approach to renormalization, a single step of renormalization procedure consists of
a functional integration of high-momentum fluctuations, followed by a rescaling of physical lengths and momenta, and
renormalization of fields. All of this leaves the non-perturbed theory unchanged, affecting only the couplings.
Before the rescaling and renormalization operations we are dealing with the so-called Wilsonian effective action ($S_{\text{eff}}$).
It describes the behaviour of fields for the processes below certain energy scale $b\Lambda$, lower than original cutoff $\Lambda$.
$S_{\text{eff}}$ will generally contain the operators of all the higer dimensions in fields and derivatives.
These corrections (...) but they allow us to neglect field modes larger than $\mu = b\Lambda$ and deal only with non-divergent diagrams.

However, $S_{\text{eff}}$ is still a quantum action, in the sense that the functional integral is performed over it. It also
in some sense lose some information about the high-energy physics.
There is another object (...) quantum effective action

% \section{Intro}

% aa

\end{document}
