\documentclass[11pt, a4paper]{article} 
% \usepackage[MeX]{polski} %polskie znaki
\usepackage{booktabs}
\usepackage{comment} 
\usepackage[utf8]{inputenc} % odpowiednie kodowanie znaków
\usepackage[T1, T2A]{fontenc} 
\usepackage{graphicx} %wstawianie obrazków
\usepackage{float} %ustawianie obraków/tabel
\usepackage{multirow}
\usepackage{amsmath, amsfonts,amsthm} 
\usepackage{amsthm} %do twierdzen itd
\usepackage{mathtools}
\usepackage{blindtext} 
%\usepackage{har2nat} %grzebie w bibliografii
\usepackage{geometry}
\usepackage{amssymb}
\usepackage{tikz-cd}
\usepackage{lscape}
\usepackage{hyperref}
\usepackage{XCharter}
\usepackage{dsfont}
\usepackage[labelfont=bf]{caption}
\usepackage{caption}
\usepackage{subcaption}
\usepackage{pbox}
\usepackage{tikz}
\usepackage{tikz-feynman}

\setlength{\parindent}{15pt} 


\def\fR{\mathbb{R}}
\def\fC{\mathbb{C}}
\def\fN{\mathbb{N}}
\def\fZ{\mathbb{Z}}
%pogrubiony symbol wektora
\def\bv{\boldsymbol{v}}
\def\bu{\boldsymbol{u}}
\def\bR{\boldsymbol{R}}
\def\id{\mathds{1}}
\def\H{\mathcal{H}}
\def\idA{\vec{\mathbb{I}}_A{} }

\def\ra{\rangle}
\def\la{\langle}

%%%%%%%%%%%%%%%%% ME new 

\definecolor{dg}{rgb}{0,.5,0}

%%%%%%%%%%%%%%%%%%%%%%%%


\newtheorem{thm}{Theorem}[section]
\newtheorem{lem}[thm]{Lemma}
\newtheorem{corol}[thm]{Corollary}
\newtheorem{defn}[thm]{Definition}
\newtheorem{rmk}{Remark}

\title{\vspace{-2cm}On the Landau pole in quantum electrodynamics and the possible quantum gravity corrections}
\author{{Wojciech Śmiałek}\\
\\
{\textit{Supervisor}} \\
{Jan Kwapisz phd.}}
\date{}

\begin{document}
\maketitle

\section*{Introduction}

\part{Gravitational corrections in the Landau pole problem - preliminaries and the review of existing results}

% \section{Renormalization and the running of couplings}

\section{Running of the gauge coupling in QED}

% Quantum Electrodynamics is famously one of the most stringently tested theories in physics. 
% Given only two input
% parameters - lepton mass and charge, it is able to describe a range of experimental data and predict otherwise
% unexpected phenomena, such as Casimir effect. 
Pure Quantum Electrodynamics is a theory of Dirac spinor field interacting with a locally $U(1)$-symmetric vector field.
Bare action of the QED reads
\begin{equation}
    S_{QED} = \int d^4 x \left( -\frac{1}{4}F_{\mu\nu}F^{\mu\nu} + \bar{\psi}(i \gamma^\mu \partial_\mu - m)\psi - e j^\mu \psi A_\mu \right)
\end{equation}
Where $j^\mu = \bar \psi \gamma^\mu \psi$ is the $U(1)$ Noether's conserved current.
\textcolor{red}{(Maybe write bare vs renormalized lagrangain and discuss how symmetry reduces number of divergent quantities to 3)}
In the perturbative QED, only three subdiagrams are responsible for any divergences encountered in the non-renormalized
theory. In the standard on-shell renormalization, one imposes in the rest frame the requirement of electron mass and charge
equal to the measured one and of normalization of kinetic terms equal to one. This suffices to render all
physical amplitudes finite and lends QED its predictive power, that has been experimentally tested
to the remarkable degree of precision.
We will focus here on the renormalization and momentum dependece of the photon propagator.
At the tree level, the expression for the photon propagator is
\begin{equation*}
\begin{aligned}
\feynmandiagram [horizontal=a to b] {
a -- [photon] b
}; \quad = D_0^{\mu\nu}(k) = \quad \frac{1}{Z_A} \frac{-i \left( g^{\mu\nu} - \frac{k^\mu k^\nu}{k^2} \right)}{k^2+i\epsilon} + \frac{-i \xi \frac{k^\mu k^\nu}{k^2}}{k^2+i\epsilon}
\end{aligned}
\end{equation*}
Where $\xi$ is a gauge fixing parameter. The $\xi$-dependent part of the propagator is not affected by loop corrections
and we will work in the Landau gauge, setting it to zero. 
Correction to the propagator in given order is computed by summing all relevant one-particle irreducible diagrams
and then summing the series of strings of such diagram sums. In the one loop approximation we can represent this diagramatically as

\begin{equation*}
    \feynmandiagram [layered layout, horizontal=a to c, node distance=0.25cm] {
    a -- [photon] b [blob] -- [photon] c
    };
    = 
    \feynmandiagram [layered layout, horizontal=a to c, node distance=0.5cm] {
    a -- [photon] c
    };
    +
    \feynmandiagram [layered layout, horizontal=a to c, node distance=0.25cm] {
        a -- [photon] b [blob] -- [photon] c
        };
    + \ldots
\end{equation*}
Let us denote the sum of all relevant 1PI diagrams as $i \Pi^{\mu\nu}(k)$ and the exact propagator as $D^{\mu\nu}(k)$. 
We can write:
\begin{eqnarray}
    D = D_0 + D_0 i \Pi D_0 + D_0 i \Pi D_0 \Pi D_0 + \ldots\\
    =  D_0 \left( 1 + i \Pi ( D_0 + D_0 i \Pi D_0 + \ldots) \right) = D_0 (1 + i\Pi D)\\
    \implies i(D^{-1})^{\mu\nu}(k) = i(D_0^{-1})^{\mu\nu}(k) + \Pi^{\mu\nu}(k)
\end{eqnarray}
So the correction of the propagator is determined only by the sum of 1PI diagrams.
Via the analysis of the possible tensor structures and Ward identity \textcolor{red}{(elaborate)}
the only possible form of $\Pi$ is
\begin{equation}
    \Pi^{\mu\nu}(k) = k^2 \varPi(k^2) \left(k^2 g^{\mu\nu} - k^\mu k^\nu \right)
\end{equation}
Now the exact propagator reads \textcolor{red}{(write about inverses and stuff)}
\begin{equation}
    D^{\mu\nu}(k) = \frac{1}{Z_A - \varPi(k^2)} \frac{-i \left( g^{\mu\nu} - \frac{k^\mu k^\nu}{k^2} \right)}{k^2+i\epsilon}
\end{equation}

% QED is a perturbatively renormalizable theory, i.e. it allows to calculate probability amplitudes for physical processes
% up to an arbitrary order in couplings.
%% One loop beta function of qed
%% Discussion of landau pole
%% General form in the presence of quantum gravity
% Wilczek paper
%% Qualitative difference

\section{The renormalization group}
[To be corrected and expanded]\\
Through the contributions of Feynman diagrams beyond the leading order, a nontrivial dependence on the energy scale
of the process can enter the Green's functions in certain interacting quantum field theories. As a simple example
(photon propagator...)(eq). Physically, this phenomenon can be interpreted as quantum field fluctuations causing
the observed physical constants, like the electron mass and charge, to change with the scale of momenta involved
in the process.\\% from which observations have been extracted.\\

(O tym ze nakladamy warunki ren. przy skali mu rownej skali procesu i ze wtedy
zaleznosc od pedu przesuwamy z loop integrals do tree level (czyli effective coupling) i jak wtedy powstaja funkcje beta. Potem Wilsonian RG cale)


\section{Quantized gravity}
% Płomienny wstęp

To incorporate gravity into the framework of quantum field theory, %which appears as the correct fundamental description of all other phenomena in nature, 
we wish to define an action for the field responsible for gravitational interaction.
By simply analyzing possible form of Lorentz-covariant linear field equations for a carrier of long-distance, 
universally attractive interactions, it is possible to deduct, that the corresponding field $\xi$ would have to be that of a massless boson with even integer spin [ref].
The simplest case of $s=0$, an analogue of Newtonian field, cannot account for the gravitational deflection of light.
The interaction of $\xi$ with matter would be via term in action of the form $\xi T^\mu_\mu$, where $T$ is a stress-energy tensor.
Stress-energy tensor of electromagnetic field is traceless and it would not interact with scalar $\xi$, contradicting known observations.
The next simplest possibility is that of $s=2$. Then the field would be a symmetric two-tensor $\xi_{\mu\nu}$. This meets our expectations as a
candidate for the quantized metric field $g_{\mu\nu}$, a fundamental object in Einstein's theory of gravity.

Continuing ... one arrives at the Fierz-Pauli equations and the action
\begin{equation}
    S_{FP} = \int d^4 x \left( -\frac{1}{2} \partial_\sigma \xi_{\mu\nu} \partial^\sigma \xi^{\mu\nu} +  \partial_\sigma \xi_\mu^\sigma \partial_\rho \xi^{\mu\rho} - \partial_\sigma \xi_\mu^\sigma \partial^\mu \xi_\rho^\rho +\frac{1}{2} \partial_\sigma \xi_\rho^\rho \partial^\sigma \xi_\rho^\rho \right)
    \label{SFP}
\end{equation}
This line of reasoning may even be an alternative way of deriving the Einstein's Field Equations. The method for recovering full, non-linear
field equations from the free massless spin-2 field equations, after accounting for the graviton self-interaction, have been shown by Deser [ref]. 
We can split metric field into flat background plus a fluctuation field and write the action yielding full Einstein's Field Equations.
\begin{gather}
    S_{EH} = \frac{1}{2 \kappa} \int d^4 x \sqrt{-\det{g}} \ R 
    \label{SEH} \\
    g_{\mu\nu} = \eta_{\mu\nu} + h_{\mu\nu}
\end{gather}
Where $R$ is the scalar curvature and $\kappa = \sqrt{8 \pi G}$ is the Einstein's gravitational constant. Expanding metric determinant and curvature to the second order in $h$ field, we arrive at the action proportional to $S_{FP}$.
At this point, we can identify $2\kappa\xi_{\mu\nu}$ with metric fluctuation field $h_{\mu\nu}$ and the two actions agree exactly.
The Einstein - Hilbert action (\ref{SEH}) should be taken as the action of the full interacting theory.

% Gravitons (?) - plane wave solutions, polarizations, gravitational waves
\subsection*{\centering Perturbative nonrenormalizability of the genereric quantum gravity}

\section{Gravitational corrections as solution to the Landau Pole / triviality problem - discussion of the existing results}

\part{Functional renormalization group approach and the presentation of results}

\section{The Functional Renormalization Group Equation}

\subsection*{\centering Effective action}

In the traditional Wilsonian approach to renormalization, a single step of renormalization procedure consists of
a functional integration of high-momentum fluctuations, followed by a rescaling of physical lengths and momenta, and
renormalization of fields. All of this leaves the non-perturbed theory unchanged, affecting only the couplings.
Before the rescaling and renormalization operations [we are dealing with] the so-called Wilsonian effective action ($S_{\text{eff}}$).
It describes the behaviour of fields for the processes below certain energy scale $b\Lambda$, lower than the original cutoff $\Lambda$.
$S_{\text{eff}}$ generally contains all operators with higer dimensions in fields and derivatives.
These corrections (...) but they allow us to neglect field modes larger than $\mu = b\Lambda$ and deal only with non-divergent diagrams.
The argument of $S_{\text{eff}}$ is still a quantum field, in the sense that the functional integral is performed over it. %It also,
% in some sense, loses information about the high-energy physics.

The object, that we will call an effective action $\Gamma$ is different and should not be confused with $S_{\text{eff}}$.
Let us start from the euclidean partition function for scalar field theory. The definition for other theories
come as a straight-forward generalization.
\begin{equation}
    Z[j] = \int \mathcal{D}\phi \ e^{-S[\phi] + \int dx j \phi}
\end{equation}
The generating functional of connected Green's functions is defined as
\begin{equation}
    W[j] = \log{Z[j]}
\end{equation}
The effective action functional is defined using the Legendre transformation of $W[j]$.
\begin{equation}
    \Gamma[\phi_c] = W[j_\phi] - \int d^4 x j_\phi(x) \phi(x)
\end{equation}
The two fields $\phi_c$ and $j_\phi$ are inverses of each other, defined as the solution to
\begin{equation}
    \phi_c(x) = \langle \hat\phi (x) \rangle_j = \frac{\delta W[j]}{\delta j(x)}
\end{equation}
The argument of the effective action is a classical field and there is no functional integral to be performed over it.
Rather, in $\Gamma$ all of the fluctuations are integrated out, but only one-particle irreducible diagrams
are included. $\Gamma$ acts as a generating functional of 1PI Green's functions. Extremizing effective, rather than
the clasical action, yields the equations of motion for vacuum expectation values of the quantum fields.

In its bare form, effective action is ill-defined, as was expected. One option is to introduce a UV cutoff $\Lambda$
and study the rg flow through divergences proportional to $\Lambda$. The modification we will employ, however, involves
an IR cutoff inserted through adding a regulator term $\Delta S_k[\phi]$ to the bare action $S[\phi]$ in the definition of partition function
and subtracted from the final form of effective action. Explicitly, this new object, called the effective average action (EAA) is defined as
\begin{gather}
    W_k[j] = \log{\int \mathcal{D}\phi \ e^{-S[\phi] - \Delta S_k[\phi] + \int dx j \phi}}\\
    \Gamma_k[\phi_c] = W_k[j_\phi] - \int d^4 x j_\phi(x) \phi(x) - \Delta S_k[\phi]
\end{gather}
Motivation for introducing EAA will become clear when we study the functional renormalization group

\subsection*{\centering Infrared regulator and the scheme dependence}

\subsection*{\centering Beta functional and the functional renormalization group}

The $\Gamma_k$ is IR-regulated, but still it is ill-defined, because of the UV divergences. 
However, in studying the scale dependence of couplings we will not use full EAA, 
but its derivative with respect to $t = \log{k}$.
We assume the theory space in which $\Gamma_k$ takes the form
\begin{equation}
    \Gamma_k = \sum_i g_i(k) \ \mathcal{O}_i [\phi]
\end{equation}
Where $\mathcal{O}_i (\phi)$ are integrals of monomials of fields or positive powers of field derivatives 
and $g_i(k)$ are scale-dependent couplings.
The coefficients in EAA derivative with respect to $t$ are therefore simply the beta functions of corresponding operators
\begin{equation}
    \frac{d \Gamma_k}{dt} = \sum_i \frac{d g_i}{dt} \ \mathcal{O}_i [\phi] = \beta_i(g,k) \ \mathcal{O}_i [\phi]
\end{equation}
The beta functions may depend on all the couplings, as well as the renormalization scale $k$.
They can be extracted from $\frac{d \Gamma_k}{dt}$ via a suitable projection operator. 
The $\frac{d \Gamma_k}{dt}$ is called the beta functional. This functional, as can be shown, is finite.
This is because the beta functional can be viewed as a difference between effective actions with infinitesimally
different cutoffs. The UV divergences in the difference will cancel, and what remains is the finite rest
dependent on the degrees of freedom with momenta close to the renormalization scale.

Let us calculate the derivatives of $W_k$ and $\Delta S_k$ with respect to $t$
\begin{gather}
    \frac{d W_k}{dt} = \frac{d}{dt}\log{Z_k} = - \frac{1}{Z_k} \int \mathcal{D}\phi \ e^{-S[\phi] - \Delta S_k[\phi] + \int dx j \phi}  \cdot \frac{d \Delta S_k}{dt} \\
    \frac{d \Delta S_k}{dt} = \frac{1}{2} \int d^4 x \ \phi \ \frac{d R_k}{dt} \ \phi
\end{gather}
This lets us write
\begin{equation}
    \frac{d \Gamma_k}{dt} = \frac{d \langle \Delta S_k \rangle}{dt} - \frac{d \Delta S_k }{dt} = \frac{1}{2} \operatorname{Tr} \left[ (\langle\phi\phi\rangle - \langle\phi\rangle^2) \cdot \frac{d R_k}{dt} \right]
\end{equation}
Where $\operatorname{Tr}$ denotes ... and the $\langle\cdots\rangle$ - ...
The expression $(\langle\phi\phi\rangle - \langle\phi\rangle^2)$ can be shown to be equal to $\frac{\delta^2 W_k}{\delta j \delta j}$ [cite]
From there, if we would express $\frac{\delta^2 W_k}{\delta j \delta j}$ in terms of $\Gamma_k$, we could
write an exact, first order differential equation for the effective average action.
In fact, the relationship between (them) is very simple. Recall, that $\Gamma_k + \Delta S_k$ is a Legendre transform of $W_k$. For any
two functions $f$ and $g$, one being the Legendre transform of the other, we have $f'' = (g'')^{-1}$. This remains true for the functional derivation.
Using this information and immediatly performing field derivative over $\Delta S_k$, we can write the equation for $\Gamma_k$:
\begin{equation}
    \frac{d \Gamma_k}{dt} = \frac{1}{2} \operatorname{Tr} \left[ \left(\frac{\delta^2 \Gamma_k}{\delta \phi \delta \phi} + R_k\right)^{-1} \cdot \frac{d R_k}{dt} \right]
    \label{FRGE}
\end{equation}
This is the functional renormalization group equation (FRGE) or the Wetterich equation. (...)
In its original form, FRGE is not well suited for performing specific calculations. One very intuitive method, which
we will use is the $\mathcal{PF}$-expansion, that allows us to use the Feynman diagrams for calculating the RHS of equation (\ref{FRGE})
up to the desired order in couplings. The term inside the trace including Second derivative of EAA
will in general be, for spinor or tensor fields, a functional hessian matrix. We can decompose this term into a
regulated inverse propagator matrix $\mathcal{P}$ and a rest, which will include the derivatives of terms non quadratic in fields.
\begin{equation}
    \frac{\delta^2 \Gamma_k}{\delta \phi \delta \phi} + R_k = \mathcal{P} + \mathcal{F}
\end{equation}
First, let us notice that the entire expression inside trace can be expressed as a $\log{(\mathcal{P}+\mathcal{F})}$, upon which acts
a $t$-derivative sensitive only on the $t$ dependence in $R_k$. Explicitly, we can write:
\begin{equation}
    \left(\mathcal{P} + \mathcal{F}\right)^{-1} \cdot \partial_t R_k = \left(\mathcal{P} + \mathcal{F}\right)^{-1} \cdot \widetilde{\partial_t} \left(\mathcal{P}+\mathcal{F}\right) = \widetilde{\partial_t} \log{\left(\mathcal{P}+\mathcal{F}\right)}; \quad \widetilde{\partial_t} = \int \partial_t R_k \frac{\delta}{\delta R_k}
\end{equation}
Now, we can recall the series expansion of $\log{(1+x)}$ around $x=0$ and after some simple manipulations, obtain an expansion of functional trace in (\ref{FRGE}) in
the number of $\mathcal{F}$-terms
\begin{gather}
    \frac{d \Gamma_k}{dt} = \frac{1}{2} \operatorname{Tr} \left[ \widetilde{\partial_t} \log{\left(\mathcal{P}+\mathcal{F}\right)} \right] = \frac{1}{2} \operatorname{Tr} \left[ \widetilde{\partial_t} \left(\log{\mathcal{P}} + \log{(1+\mathcal{P}^{-1}\mathcal{F})}\right) \right] \\ =  \frac{1}{2} \operatorname{Tr} \left[ \widetilde{\partial_t} \log{\mathcal{P}} \right] + \frac{1}{2} \sum_{n=1}^{\infty} \frac{(-1)^{n-1}}{n} \operatorname{Tr}\left[\widetilde{\partial_t}\left(\mathcal{P}^{-1}\mathcal{F}\right)^n\right]
\end{gather}


\section{Framework - quantum einstein gravity coupled to the $U(1)$ gauge theory}

\section{Results (...)}

\section*{Summary}

\end{document}
