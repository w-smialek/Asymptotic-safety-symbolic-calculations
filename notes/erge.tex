\documentclass[11pt, a4paper]{article} 
\usepackage[MeX]{polski} %polskie znaki
\usepackage{booktabs}
\usepackage{comment} 
\usepackage[utf8]{inputenc} % odpowiednie kodowanie znaków
\usepackage[T1, T2A]{fontenc} 
\usepackage{graphicx} %wstawianie obrazków
\usepackage{float} %ustawianie obraków/tabel
\usepackage{multirow}
\usepackage{amsmath, amsfonts,amsthm} 
\usepackage{amsthm} %do twierdzen itd
\usepackage{mathtools}
\usepackage{blindtext} 
%\usepackage{har2nat} %grzebie w bibliografii
\usepackage{geometry}
\usepackage{amssymb}
\usepackage{tikz-cd}
\usepackage{lscape}
\usepackage{hyperref}
\usepackage{XCharter}
\usepackage{dsfont}
\usepackage[labelfont=bf]{caption}
\usepackage{caption}
\usepackage{subcaption}
\usepackage{pbox}

\setlength{\parindent}{15pt} 


\def\fR{\mathbb{R}}
\def\fC{\mathbb{C}}
\def\fN{\mathbb{N}}
\def\fZ{\mathbb{Z}}
%pogrubiony symbol wektora
\def\bv{\boldsymbol{v}}
\def\bu{\boldsymbol{u}}
\def\bR{\boldsymbol{R}}
\def\id{\mathds{1}}
\def\H{\mathcal{H}}
\def\idA{\vec{\mathbb{I}}_A{} }

\def\ra{\rangle}
\def\la{\langle}

%%%%%%%%%%%%%%%%% ME new 

\definecolor{dg}{rgb}{0,.5,0}

%%%%%%%%%%%%%%%%%%%%%%%%


\newtheorem{thm}{Theorem}[section]
\newtheorem{lem}[thm]{Lemma}
\newtheorem{corol}[thm]{Corollary}
\newtheorem{defn}[thm]{Definition}
\newtheorem{rmk}{Remark}

\title{\vspace{-2cm}The Functional Renormalization Group Equation}
% \author{Rafał Bistroń}
\date{}

\begin{document}

\maketitle

% \tableofcontents

\subsection*{\centering Effective action}

In the traditional Wilsonian approach to renormalization, a single step of renormalization procedure consists of
a functional integration of high-momentum fluctuations, followed by a rescaling of physical lengths and momenta, and
renormalization of fields. All of this leaves the non-perturbed theory unchanged, affecting only the couplings.
Before the rescaling and renormalization operations [we are dealing with] the so-called Wilsonian effective action ($S_{\text{eff}}$).
It describes the behaviour of fields for the processes below certain energy scale $b\Lambda$, lower than the original cutoff $\Lambda$.
$S_{\text{eff}}$ generally contains all operators with higer dimensions in fields and derivatives.
These corrections (...) but they allow us to neglect field modes larger than $\mu = b\Lambda$ and deal only with non-divergent diagrams.
The argument of $S_{\text{eff}}$ is still a quantum field, in the sense that the functional integral is performed over it. %It also,
% in some sense, loses information about the high-energy physics.

The object, that we will call an effective action $\Gamma$ is different and should not be confused with $S_{\text{eff}}$.
Let us start from the euclidean partition function for scalar field theory. The definition for other theories
come as a straight-forward generalization.
\begin{equation}
    Z[j] = \int \mathcal{D}\phi \ e^{-S[\phi] + \int dx j \phi}
\end{equation}
The generating functional of connected Green's functions is defined as
\begin{equation}
    W[j] = \log{Z[j]}
\end{equation}
The effective action functional is defined using the Legendre transformation of $W[j]$.
\begin{equation}
    \Gamma[\phi_c] = W[j_\phi] - \int d^4 x j_\phi(x) \phi(x)
\end{equation}
The two fields $\phi_c$ and $j_\phi$, which are inverses of each other, are defined as the solution to
\begin{equation}
    \phi_c(x) = \langle \hat\phi (x) \rangle_j = \frac{\delta W[j]}{\delta j(x)}
\end{equation}
The argument of the effective action is a classical field and there is no functional integral to be performed over it.
Rather, in $\Gamma$ all of the fluctuations are integrated out, but only one-particle irreducible diagrams
are included. $\Gamma$ acts as a generating functional of 1PI Green's functions. Extremizing effective, rather than
the clasical action, yields the equations of motion for vacuum expectation values of the quantum fields.

In its bare form, effective action is ill-defined, as was expected. One option is to introduce a UV cutoff $\Lambda$
and study the rg flow through divergences proportional to $\Lambda$. The modification we will employ, however, involves
an IR cutoff inserted through adding a regulator term $\Delta S_k[\phi]$ to the bare action $S[\phi]$ in the definition of partition function
and subtracted from the final form of effective action. Explicitly, this new object, called the effective average action (EAA) is defined as
\begin{eqnarray}
    W_k[j] = \log{\int \mathcal{D}\phi \ e^{-S[\phi] - \Delta S_k[\phi] + \int dx j \phi}}\\
    \Gamma_k[\phi_c] = W_k[j_\phi] - \int d^4 x j_\phi(x) \phi(x) - \Delta S_k[\phi]
\end{eqnarray}
Motivation for introducing EAA will become clear when we study the functional renormalization group

\subsection*{\centering Infrared regulator and the scheme dependence}

\subsection*{\centering Beta functional and the functional renormalization group}

The $\Gamma_k$ is IR-regulated, but still it is ill-defined, because of the UV divergences. 
However, in studying the scale dependence of couplings we will not use full EAA, 
but its derivative with respect to $t = \log{k}$.
We assume the theory space in which $\Gamma_k$ takes the form
\begin{equation}
    \Gamma_k = \sum_i g_i(k) \ \mathcal{O}_i [\phi]
\end{equation}
Where $\mathcal{O}_i (\phi)$ are integrals of monomials of fields or positive powers of field derivatives 
and $g_i(k)$ are scale-dependent couplings.
The coefficients in EAA derivative with respect to $t$ are therefore simply the beta functions of corresponding operators
\begin{equation}
    \frac{d \Gamma_k}{dt} = \sum_i \frac{d g_i}{dt} \ \mathcal{O}_i [\phi] = \beta_i(g,k) \ \mathcal{O}_i [\phi]
\end{equation}
The beta functions may depend on all the couplings, as well as the renormalization scale $k$.
They can be extracted from $\frac{d \Gamma_k}{dt}$ via a suitable projection operator. 
The $\frac{d \Gamma_k}{dt}$ is called the beta functional. This functional, as can be shown, is finite.
This is because the beta functional can be viewed as a difference between effective actions with infinitesimal
difference in cutoff. The UV divergences in the difference will cancel, and what remains is the finite rest
dependent on the degrees of freedom with momenta close to the renormalization scale.

Ordinary effective action can be split into the tree-level, one loop and higher order parts
\begin{equation}
    \Gamma = S + \frac{1}{2} \operatorname{Tr} \log{\frac{\delta^2 S}{\delta \phi \delta \phi}} + \left( \pbox{40mm}{\ \ \ \ \ \ Remaining \\ Feynman diagrams} \right)
\end{equation}
The same can be done for the EAA, as the only difference between the two is replacement of the action by its regulated version and
the subtraction of the regulator at the end. The one-loop approximation reads
\begin{equation}
    \Gamma_k^{(1)} = S + \frac{1}{2} \operatorname{Tr} \log{\frac{\delta^2 (S+\Delta S_k)}{\delta \phi \delta \phi}} = S + \frac{1}{2} \operatorname{Tr} \log{\left(\frac{\delta^2 S}{\delta \phi \delta \phi} + R_k\right)}
\end{equation}
%% Inaczej - to nie podręcznik. Od razu wyprowadzać ścisłe równanie


\end{document}
